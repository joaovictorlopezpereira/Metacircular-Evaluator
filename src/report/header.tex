%->8---------------Language-and-Document-Packages---------------8<-%
\usepackage[english]{babel}
\usepackage[utf8]{inputenc}
\usepackage[T1]{fontenc}
%->8------------------------------------------------------------8<-%

%->8--------Referencing-and-Figures-Packages-and-Commands-------8<-%
\usepackage{tocloft}
\usepackage{hyperref}
\usepackage{csquotes}
\usepackage{graphicx}
\usepackage{float}
\usepackage[backend=biber]{biblatex}
\addbibresource{repertoire.bib}
\graphicspath{{figures/}}
%->8------------------------------------------------------------8<-%

%->8----------Formatting-Related-Packages-and-Commands----------8<-%
\usepackage[margin=1in]{geometry}
\usepackage{amsmath, amssymb, amsthm, cancel}
\usepackage{upquote}
\usepackage{indentfirst}
\usepackage{listings}
\usepackage{xcolor}
\setlength{\parskip}{\baselineskip}
\setlength{\footnotesep}{0.8em}
\setlength{\skip\footins}{2em}
\setlength{\cftbeforechapskip}{12pt}
\setlength{\cftbeforesecskip}{6pt}
%->8------------------------------------------------------------8<-%

%->8---------------------Math-Related-Macros--------------------8<-%
\newtheorem{theorem}{Teorema}[section]
\newcommand{\demonstracao}{\noindent\textbf{Demonstração: }}
\newcommand{\corolario}{\noindent\textbf{Corolário: }}
\renewcommand{\qed}{\hfill\square}
\newcommand{\norm}[1]{\left\lVert#1\right\rVert}
\newcommand{\abs}[1]{\left\lvert#1\right\rvert}
\newcommand{\dotprod}[1]{\left\langle#1\right\rangle}
\newcommand{\proj}[2]{\text{proj}_{#1}(#2)}
\renewcommand{\min}[1]{\text{Min}_{#1}\,\,}
\newcommand{\argmin}[1]{\text{Argmin}_{#1}\,\,}
\newcommand{\argmax}[1]{\text{Argmin}_{#1}\,\,}
\newcommand{\mul}{\cdot}
\newcommand{\grad}{\nabla}
\renewcommand{\deg}{\ensuremath{^\circ}}
% \renewcommand{\sin}{\text{sen}}
\newenvironment{jmatrix}{\begin{bmatrix}}{\end{bmatrix}}
%->8------------------------------------------------------------8<-%

%->8---------------------Code-Related-Macros--------------------8<-%
\lstnewenvironment{code}[1][]{
  \lstset{
    basicstyle=\ttfamily,
    columns=flexible,
    breaklines=true,
    breakatwhitespace=true,
    frame=none,
    basewidth=0.5em,
    aboveskip=13pt,
    belowskip=0pt,
    #1
  }
}{}
\lstnewenvironment{styledoutput}[1][]{
  \lstset{
    basicstyle=\ttfamily\itshape,
    breaklines=true,
    breakatwhitespace=true,
    frame=none,
    aboveskip=0pt,
    #1
  }
}{}
\newcommand{\userinput}[1]{\noindent\texttt{\textit{>}}\texttt{#1}}
\newcommand{\useroutput}[1]{\textit{\texttt{#1}}}
\newcommand{\var}[1]{\texttt{#1}}
%->8------------------------------------------------------------8<-%

%->8------------------Formatting-Related-Macros-----------------8<-%
\newcommand{\excerpt}[4]{ %{phrase}{author}{font}{cite}
  \vspace{25pt}
  \begin{quote}
    \begin{center}
      \hfill
      \begin{minipage}{0.65\textwidth}
        #1 #4 \\[10pt] #2, \\ \textit{#3}
      \end{minipage}
    \end{center}
  \end{quote}
  \vspace{25pt}
}
\newcommand{\excerptinfn}[4]{\textit{``#1''} #2, \textit{#3}#4}
\newcommand{\todo}[1]{\textcolor{red}{\textbf{Todo:} #1}}
\newcommand{\nonport}[1]{\textit{#1}}
%->8------------------------------------------------------------8<-%
